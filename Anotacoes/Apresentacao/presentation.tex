% Full instructions available at:
% https://github.com/elauksap/focus-beamertheme

\documentclass{beamer}
\usetheme{focus}
\usepackage{graphicx,subcaption}
\usepackage[utf8]{inputenc}

\title{Formulação Corrente-Vorticidade \\ e Transporte de Calor}
\subtitle{MEF com Malhas Não Estruturadas}
\author{\textbf{L. H. Carnevale} \\ G. R. Anjos}
\titlegraphic{\begin{center}
	\hspace{3cm}\includegraphics[scale=0.3]{fig/uerj.png}
	\end{center}}
\date{30/07/2018}


%------------------------------------------------


\begin{document}
    \begin{frame}
        \maketitle
    \end{frame}
    
    
    \begin{frame}{Sumário}
		\tableofcontents
	\end{frame}
    
%------------------------------------------------
    
	\section{Equações e Método}	
	\subsection{Motivação}	
	\begin{frame}{Motivação}
		  Arrefecimento simultâneo de microprocessadores empilhados\footnote{figures from Prof. John Thome, Laboratoire de Transfert de Chaleur et de Masse (LTCM), EPFL, \textit{https://ltcm.epfl.ch}} com diferentes geometrias
        \begin{figure}
        	\includegraphics[scale=2.2]{fig/CMOSAIC.png}
        	\includegraphics[scale=0.6]{fig/channels_67.jpg}
        \end{figure}
    \end{frame}

%------------------------------------------------


	\subsection{Equações}
    \begin{frame}{Equações Gerais}
    	\begin{columns}			
			\begin{column}{0.5\textwidth}
			   	\begin{block}{\small{Corrente-Vorticidade}}
			   	 	\begin{equation}			   	 	
\frac{\partial \omega_z}{\partial t} + \mathbf{v} \cdot \nabla \omega_z \ = \frac{1}{Re} \nabla^2 \omega_z		
		        	\end{equation}
        	       	\begin{equation}
\frac{\partial^2 \psi}{\partial x^2} +\frac{\partial^2 \psi}{\partial y^2} = \omega_z
		        	\end{equation}  
                	\begin{equation}
\frac{\partial \psi}{\partial y} = v_x \  , \ \ \frac{\partial \psi}{\partial x} = -v_y        
        			\end{equation}        
	        	\end{block}
        	\end{column}
			\begin{column}{0.5\textwidth}
	    		\begin{block}{\small{Transporte de Calor}}
    	    	  	\begin{equation}	
\frac{\partial T}{\partial t} + \mathbf{v} \cdot \nabla T  \ = \frac{1}{Re Pr} \nabla^2 T 	 	
		 			\end{equation}
		 		\end{block}
				\begin{block}{\small{Condições de Contorno e Inicias}}
					\begin{equation}
				\omega_z = \frac{\partial v_y}{\partial x}-\frac{\partial v_x}{\partial y}
					\end{equation}
					\begin{equation}
				\psi = \psi_0
					\end{equation}					
				\end{block}	 	
			\end{column}
		\end{columns}	 	
	 \end{frame}
    
%------------------------------------------------
    
    \subsection{Formulação MEF}
    \begin{frame}{Formulação MEF}
		\begin{columns}
			\begin{column}{0.5\textwidth}
				\begin{figure}[H]
					\includegraphics[scale=0.23]{fig/triang.png}		
				\end{figure}
				\begin{block}{\small{Coeficientes da Função de Forma}}
			\footnotesize{$a_i = x_j y_k - x_k y_j$ ; $b_i = y_j - y_k$ ; $c_i = x_k - x_j$} \\
			\footnotesize{$a_j = x_k y_i - x_i y_k$ ; $b_j = y_k - y_i$ ; $c_j = x_i - x_k$}\\				
			\footnotesize{$a_k = x_i y_j - x_j y_i$ ; $b_k = y_i - y_j$ ; $c_k = x_j - x_i$} 				
				\end{block}
			\end{column}

			\begin{column}{0.5\textwidth}
			\footnotesize{\begin{equation}
			\mathbf{M} = \frac{A}{12}
				\begin{bmatrix}
					2 & 1 & 1   \\
					1 & 2 & 1   \\
					1 & 1 & 2  
				\end{bmatrix} \notag 
			\end{equation}

			\begin{equation}
			\mathbf{B} =
				\begin{bmatrix}
					B_x   \\
					B_y  
				\end{bmatrix} = \frac{1}{2A}
				\begin{bmatrix}
					b_i & b_j & b_k   \\
					c_i & c_j & c_k   
				\end{bmatrix}\notag 
			\end{equation}

			\begin{equation}
			\mathbf{G_x} = \frac{1}{6}
				\begin{bmatrix}
					b_i & b_j & b_k   \\
					b_i & b_j & b_k   \\
					b_i & b_j & b_k   
					\end{bmatrix}\notag
			\end{equation}

			\begin{equation}
			\mathbf{G_y} = \frac{1}{6}
				\begin{bmatrix}
					c_i & c_j & c_k    \\
					c_i & c_j & c_k    \\
					c_i & c_j & c_k    
					\end{bmatrix}\notag
			\end{equation}			

			\begin{equation}
			\mathbf{K} = \mathbf{B}^T\mathbf{B} \notag 
			\end{equation}	

			\begin{equation}
			A = a_i +a_j+a_k \notag 
			\end{equation}			}		

			\end{column}
		\end{columns}     
    \end{frame}

%------------------------------------------------

    \begin{frame}{Forma Matricial}
		\begin{block}{Corrente-Vorticidade}
			\begin{equation}
				\left( \frac{\mathbf{M}}{\Delta t}  + \frac{1}{Re}\mathbf{K} +  v_x\mathbf{G_x} + v_y\mathbf{G_y}  \right) \omega_z^{n+1} = \left( \frac{\mathbf{M}}{\Delta t} \right) \omega_z^n  + \frac{\mathbf{f}}{Re} \notag 
			\end{equation}	
			\begin{equation}
				\mathbf{K}\psi = \mathbf{M} \omega_z^{n+1} + \mathbf{f}\notag 
			\end{equation}				
			\begin{equation}
				v_x = \mathbf{G_y} \psi  \texttt{  ,  }   v_y = - \mathbf{G_x}\psi\notag 
			\end{equation}
			\end{block}	
			\begin{block}{Transport de Calor}
			\begin{equation}
				\left( \frac{\mathbf{M}}{\Delta t}  + \frac{1}{RePr}\mathbf{K} +  v_x\mathbf{G_x} + v_y\mathbf{G_y}  \right) T^{n+1} = \left( \frac{\mathbf{M}}{\Delta t} \right) T^n  + \frac{\mathbf{f}}{RePr} \notag 
			\end{equation}			
		\end{block}	 	
	 \end{frame}



%------------------------------------------------
    
    \subsection{Estrutura do Algoritmo}    
    \begin{frame}{Estrutura do Algoritmo}
        
        \begin{figure}
        	\includegraphics[scale=0.6]{fig/flux.png}
        \end{figure}
    \end{frame}
    
    
    
\section{Resultados}
        
%------------------------------------------------
        
    \subsection{Escoamento de Poiseuille entre placas aquecidas}
    \begin{frame}{Escoamento de Poiseuille entre placas aquecidas}
    Caso de Validação 
		\begin{figure}[H]
    		\includegraphics[scale=0.36]{fig/poi.png}
%		    \caption{Unstructured Triangular Mesh}
		 \end{figure}
		 \vspace{-0.5cm}
		\begin{figure}[H]
    		\includegraphics[scale=0.54]{images/mesh-poi.png}
%		    \caption{Unstructured Triangular Mesh}
		 \end{figure}


	\end{frame}
   
%------------------------------------------------
   
    \begin{frame}{Escoamento de Poiseuille entre placas aquecidas}
    		 \begin{block}{Parâmetros}
			$Re = 10$ , $Pr = 1$ , $\Delta t = 0.1$ , 100 iterações		
		 \end{block}
        \begin{figure}[H]

  \begin{subfigure}[t]{.5\linewidth}
  	\centering
    \includegraphics[width=.98\linewidth]{images/temp.png}
    \caption{Temperatura}
  \end{subfigure}
  \hspace{-0.182cm}
  \begin{subfigure}[t]{.5\linewidth}
    \centering  
    \includegraphics[width=.98\linewidth]{images/vel.png}
    \caption{Velocidade}
  \end{subfigure}  

	\vspace{0.6cm}  
  
  \begin{subfigure}[t]{.5\linewidth}
   \centering
   \includegraphics[width=0.98\linewidth]{images/w-poi.png}
    \caption{Vorticidade}
  \end{subfigure}
  \hspace{-0.182cm}
  \begin{subfigure}[t]{.5\linewidth}
  	\centering
	\includegraphics[width=0.98\linewidth]{images/psi-poi.png}
    \caption{Função de Corrente}
  \end{subfigure}
	
		\label{bfs}
\end{figure}        
    \end{frame}
  
%------------------------------------------------
   
    \begin{frame}{Escoamento de Poiseuille entre placas aquecidas}
        \begin{figure}[H]
        \centering
        	\includegraphics[width=4.5cm]{images/t_graph.png} \hspace{0.2cm}
			\includegraphics[width=5.5cm]{images/v_graph.png} %\\
		
				%\hspace{1cm} (a)               \hspace{4.5cm}     (b)
%				\caption{Comparison of the numerical soluion obtaind with the stream function-vorticity finite element solver and the exact solution for (a) Velocity comparison; (b) Temperature comparison}
		\end{figure}
		
		\small{Solução analítica:}
			\begin{columns}
			\hspace{0.6cm}\begin{column}{0.5\textwidth}
         	\footnotesize{\begin{equation} 
         		T = \frac{-15}{48}q + \frac{2}{Re (-\partial P / \partial x)}qx + 3q \left(\frac{y^2}{2} - \frac{y^4}{3} \right) \notag
         	\end{equation}}
         	\end{column}
         	\begin{column}{0.5\textwidth}
\footnotesize{\begin{equation} 
         		u = 6y(1-y) \notag
         	\end{equation}}

         	\end{column}
         	\end{columns}
    \end{frame}
  
%------------------------------------------------
    
    \subsection{Transporte de calor entre placas não regulares}
    \begin{frame}{Transporte de calor entre placas não regulares}
		\begin{itemize}
		\item Condições de contorno similares ao caso anterior
		\item Fluxo de calor $q=1$ definido apenas nas regiões retangulares
		
		\item Mesmo número de Reynolds e Prandtl 
		\end{itemize}

		\begin{figure}
        	\includegraphics[scale=0.5]{fig/mesh-ru.png}
        	\caption{Malha não estruturada obtida com o software Gmsh}
        \end{figure}		
		
    \end{frame}
    
%------------------------------------------------
    
     \begin{frame}{Transporte de calor entre placas não regulares}
        \begin{figure}[H]
  \begin{subfigure}[t]{.6\linewidth}
  	\centering
    \includegraphics[width=0.98\linewidth]{images/temp_r.png}
    \caption{Temperatura}
  \end{subfigure}

  \begin{subfigure}[t]{.6\linewidth}
    \centering  
    \includegraphics[width=0.98\linewidth]{images/vel_r.png}
    \caption{Velocidade}
  \end{subfigure}  
    
  \begin{subfigure}[t]{.6\linewidth}
   \centering
   \includegraphics[width=0.98\linewidth]{images/w-ru.png}
    \caption{Vorticidade}
  \end{subfigure}

  \begin{subfigure}[t]{.6\linewidth}
  	\centering
	\includegraphics[width=0.98\linewidth]{images/psi-ru.png}
    \caption{Função Corrente}
  \end{subfigure}
	%\caption{Numerical solutions obtained by the stream function-vorticity FEM code for the complex geometry proposed}  
	\label{prop}

\end{figure}
    \end{frame}
    
%------------------------------------------------
 
    \subsection{Conclusão e Próximo Desenvolvimento}
    \begin{frame}{Conclusão e Próximo Desenvolvimento}
    \begin{itemize}
    \item Boa aplicação com a equação do transporte de calor
    \item Necessário aprimorar a metodologia para acomodar valores mais elevados de Re 
    \item Formulação é melhor aplicada à casos 2D
    \end{itemize}
%		
%		\begin{figure}
%	    	\includegraphics[width=2cm]{fig/conemmesh.png} \hspace{0.2cm}
%	    	\includegraphics[width=2cm]{fig/conem1.png}
%	    	\caption{\footnotesize{Heat transport on two solid regions} }
%    	\end{figure}

	    \begin{figure}
	    \centering
	    	\includegraphics[scale=0.2]{fig/mesh.png} \hspace{0.2cm}
	    	\includegraphics[scale=0.2]{fig/temp2.png} \hspace{0.2cm}
	    	\includegraphics[scale=0.2]{fig/vel2.png}
	    	\caption{\footnotesize{Resultados preliminares de um problema de troca de calor conjugado}}
	    \end{figure}
	 \end{frame}
            
%------------------------------------------------

    \begin{frame}%[focus]    
\centering  \huge{Obrigado!}
        \begin{figure}
        	\includegraphics[scale=0.35]{fig/uerj.png}
        \end{figure}
    \end{frame}
\end{document}
