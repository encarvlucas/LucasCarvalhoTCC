\documentclass[a4paper]{article}
\usepackage[utf8]{inputenc}
\usepackage[left=1in,right=1in,top=1in,bottom=1in]{geometry}
\usepackage{amsmath}
\usepackage{amssymb}
\usepackage{graphicx}
\usepackage{cancel}
\usepackage{mathtools}
\usepackage{placeins}
\DeclarePairedDelimiter\abs{\lvert}{\rvert}%
\DeclarePairedDelimiter\norm{\lVert}{\rVert}%

%opening
\title{Anotações}
\author{Lucas Carvalho de Sousa}

\begin{document}

\maketitle

\begin{abstract}

\end{abstract}
\newpage
\section{Equação de Transmissão de Calor}

Temos que a forma simplificada da equação geral de transmissão de calor é:

\begin{equation}
 \frac{\partial T}{\partial t}-k\nabla^2 T=Q
\end{equation}

\subsection{Discretização}

A discretização desta equação pode ser realizada por dois métodos:
\begin{list}{Método}{}
 \item[] Explícito
 \begin{equation}
  \dfrac{T^{n+1}_i-T^{n}_i}{\Delta t_{n+1}} - \dfrac{k T^{n}_{i-1}-2k T^{n}_i+k T^{n}_{i+1}}{\Delta x_i} = Q^{n}_i 
 \end{equation}
 \item[] Implícito
 \begin{equation}
  \dfrac{T^{n+1}_i-T^{n}_i}{\Delta t_{n+1}} - \dfrac{k T^{n+1}_{i-1}-2k T^{n+1}_i+k T^{n+1}_{i+1}}{\Delta x_i} = Q^{n}_i
 \end{equation}
\end{list}

\FloatBarrier
%\section{Bibliografia}
% \begin{thebibliography}{00}
% 1) R.J. Biezuner. Métodos Numéricos para Equações Parciais Elípticas. Notas de Aula, 2007.
% \\ \ \\
% 2) A.O. Fortuna. Técnicas Computacionais para Dinâmica dos Fluidos: Conceitos Básicos e Aplicações. Edusp, 2000.
% \\ \ \\
% 3) R.J. LeVeque. Finite Difference Methods for Ordinary and Partial Differential Equations. Steady-State and Time-Dependant Problems. SIAM, 2007.
% \\ \ \\
% 4) J.R. Rodrigues. Introdução à Simulação de Reservatórios Petrolíferos, Programa de Verão, LNCC 2015.
% 
% \end{thebibliography}

\end{document}
