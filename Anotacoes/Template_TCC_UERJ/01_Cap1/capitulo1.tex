\chapter{REVISÃO DA LITERATURA}
\label{cap1}
\section{\textbf{Introdução}}

Tópicos:
\textcolor{red}{  
\begin{itemize}
    \item O papel do estudo da formação de padrões na eng mec
    \item Background teórico e disciplinas/áreas necessárias
    \item Casos de interesse
\end{itemize}}

\section{\textbf{Reações-difusão em ciência dos materiais}}


% \section{\textbf{Motivação do Projeto}}

% O reconhecimento de que o fenômeno difusional está presente em muitos âmbitos de diversos problemas físicos nos leva ao questionamento de como ou porque outros fenômenos ocorrem envolvendo ou em função deste.

% \section{\textbf{Organização do Projeto}}
% Além do presente capítulo, o projeto encontra-se organizado em outros seis como descritos abaixo.\\
% \noindent\textbf{Capítulo 2}\par
% Neste capítulo \\
% \noindent\textbf{Capítulo 3}\par
% Neste capítulo são apresentados...\\
% \noindent\textbf{Capítulo 4}\par
% Neste capítulo são apresentados...\\
% \noindent\textbf{Capítulo 5}\par
% Neste capítulo são apresentados...\\
% \section{\textbf{Conclusão}}