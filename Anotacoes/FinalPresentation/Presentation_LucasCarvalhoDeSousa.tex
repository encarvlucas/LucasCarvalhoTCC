% !TEX encoding = UTF-8 Unicode
%%%%%%%%%%%%%%%%%%%%%%%%%%%%%%%%%%%%%%%%%
% Beamer Presentation
% LaTeX Template
% Version 1.0 (10/11/12)
%
% This template has been downloaded from:
% http://www.LaTeXTemplates.com
%
% License:
% CC BY-NC-SA 3.0 (http://creativecommons.org/licenses/by-nc-sa/3.0/)
%
%%%%%%%%%%%%%%%%%%%%%%%%%%%%%%%%%%%%%%%%%

%----------------------------------------------------------------------------------------
%	PACKAGES AND THEMES
%----------------------------------------------------------------------------------------

\documentclass{beamer}

\mode<presentation> {

% The Beamer class comes with a number of default slide themes
% which change the colors and layouts of slides. Below this is a list
% of all the themes, uncomment each in turn to see what they look like.

%\usetheme{default}
%\usetheme{AnnArbor}
%\usetheme{Antibes}
%\usetheme{Bergen}
%\usetheme{Berkeley}
%\usetheme{Berlin}
%\usetheme{Boadilla}
%\usetheme{CambridgeUS}
%\usetheme{Copenhagen}
%\usetheme{Darmstadt}
%\usetheme{Dresden}
\usetheme{Frankfurt}
%\usetheme{Goettingen}
%\usetheme{Hannover}
%\usetheme{Ilmenau}
%\usetheme{JuanLesPins}
%\usetheme{Luebeck}
%\usetheme{Madrid}
%\usetheme{Malmoe}
%\usetheme{Marburg}
%\usetheme{Montpellier}
%\usetheme{PaloAlto}
%\usetheme{Pittsburgh}
%\usetheme{Rochester}
%\usetheme{Singapore}
%\usetheme{Szeged}
%\usetheme{Warsaw}

% As well as themes, the Beamer class has a number of color themes
% for any slide theme. Uncomment each of these in turn to see how it
% changes the colors of your current slide theme.

%\usecolortheme{albatross}
%\usecolortheme{beaver}
%\usecolortheme{beetle}
%\usecolortheme{crane}
%\usecolortheme{dolphin}
%\usecolortheme{dove}
%\usecolortheme{fly}
%\usecolortheme{lily}
%\usecolortheme{orchid}
%\usecolortheme{rose}
%\usecolortheme{seagull}
%\usecolortheme{seahorse}
%\usecolortheme{whale}
%\usecolortheme{wolverine}

%\setbeamertemplate{footline} % To remove the footer line in all slides uncomment this line
%\setbeamertemplate{footline}[page number] % To replace the footer line in all slides with a simple slide count uncomment this line

%\setbeamertemplate{navigation symbols}{} % To remove the navigation symbols from the bottom of all slides uncomment this line
}

\usepackage[portuguese]{babel}
\usepackage[utf8]{inputenc}
\usepackage{graphicx} % Allows including images
\usepackage{booktabs} % Allows the use of \toprule, \midrule and \bottomrule in tables
\usepackage{tikz}
\usepackage{stackengine}
\usepackage{hyperref}
\usepackage{subcaption}
\usepackage{float}
\usepackage{amsmath}

% \usepackage{todonotes} % REMOVE LATER

%  NEW COMMANDS
\def\stackalignment{r}
\newcommand{\figcopyright}[4]
{
  \begin{figure}
    \stackunder{
      \includegraphics[width=0.75\linewidth]{#1}
    } {\raggedleft \tiny Fonte:\href{#2}{\textcopyright \ #3.}}
    \caption{#4}
  \end{figure}
}

\newcommand{\partfrac}[3] % Bugged function, must be called as \partfrac2
{
  \ensuremath{\dfrac{\partial{#2}}{\partial{#3}}}
}

\AtBeginSection[]{
  \begin{frame}
  \vfill
  \centering
  \begin{beamercolorbox}[sep=8pt,center,shadow=true,rounded=true]{title}
    \usebeamerfont{title}\secname\par%
  \end{beamercolorbox}
  \vfill
  \end{frame}
}

%----------------------------------------------------------------------------------------
%	TITLE PAGE
%----------------------------------------------------------------------------------------

\title[Métodos Numéricos]{Simulação Numérica De Escoamentos Dispersos Em Turbomáquinas Utilizando Método De Elementos Finitos} % The short title appears at the bottom of every slide, the full title is only on the title page

\author{\textbf{Lucas Carvalho De Sousa} \\ Gustavo Rabello Dos Anjos} % Your name
\institute[UERJ] % Your institution as it will appear on the bottom of every slide, may be shorthand to save space
{
  Universidade do Estado do Rio de Janeiro \\ % Your institution for the title page
  \medskip
  \href{mailto:encarvlucas@hotmail.com}{\textit{encarvlucas@hotmail.com}} % Your email address
}
\date{25 de Junho de 2019} %\today Date, can be changed to a custom date

\titlegraphic{%
  \makebox[0.9\paperwidth]{%
    \includegraphics[width=1.5cm,keepaspectratio]{figure/UERJ.png}%
    \hfill%
    \includegraphics[height=1.5cm,keepaspectratio]{figure/fen-new.png}%
  }%
}

\begin{document}

\begin{frame}
  \titlepage % Print the title page as the first slide
\end{frame}

\begin{frame}
  \frametitle{Sumário} % Table of contents slide, comment this block out to remove it
  \tableofcontents % Throughout your presentation, if you choose to use \section{} and \subsection{} commands, these will automatically be printed on this slide as an overview of your presentation
\end{frame}

%----------------------------------------------------------------------------------------
%	PRESENTATION SLIDES
%----------------------------------------------------------------------------------------

%----------------------------------------------------------------------------------------------------------------------------------------------------
\section{Introdução} % Sections can be created in order to organize your presentation into discrete blocks, all sections and subsections are automatically printed in the table of contents as an overview of the talk
%----------------------------------------------------------------------------------------------------------------------------------------------------

\subsection{Simulação de Escoamentos Bidimensionais com Partículas} % A subsection can be created just before a set of slides with a common theme to further break down your presentation into chunks
\begin{frame}
  \frametitle{\subsecname}
  %-Este exemplo demonstra o que é um escoamento particulado e como as partículas interagem
  
  \begin{block}{Objetivo deste trabalho:}
    Desenvolver uma biblioteca de Python para a simulação de escoamentos particulados.
  \end{block}
  
  \begin{figure}
    \stackunder{
      \includegraphics[width=\linewidth]{figure/dispersed_flow.pdf}
    } {\raggedleft \tiny Escoamento entre placas, Hagen-Poiseuille.}
  \end{figure}

\end{frame}

\subsection{Escoamentos em Turbomáquinas}
\begin{frame}
  \frametitle{\subsecname}
  %-Este exemplo demonstra que é apropriado assumir que o escoamento no disco de uma turbomáquina é
  % bidimensional, e sem efeito da gravidade
  
  \begin{block}{}
    Estudar como partículas se comportam dentro de uma turbomáquina em funcionamento.
  \end{block}
  
  \begin{figure}
    \stackunder{
      \includegraphics[width=0.6\linewidth]{figure/Washing_machine_agitator.jpg}
    } {\raggedleft \tiny Fonte:\href{https://commons.wikimedia.org/wiki/File:Washing_machine_agitator.JPG}
      {\textcopyright \ BrokenSphere / Wikimedia Commons.}}
  \end{figure}
  
\end{frame}

%----------------------------------------------------------------------------------------------------------------------------------------------------
\section{Equações de Governo}
%----------------------------------------------------------------------------------------------------------------------------------------------------

\subsection{Formulação Corrente-Vorticidade}
\begin{frame}
  \frametitle{\subsecname}
  %-Esta forma e obtida pela equação de Navier-Stoakes
  
  \begin{block}{Hipóteses tomadas}
    \begin{itemize}
     \item Fluído incompressível
     \item Fluído newtoniano
    \end{itemize}
  \end{block}
  
  \begin{block}{Equação de Navier-Stoakes}
    \centering
    $\dfrac{\partial \vec{v}_f}{\partial t} + \vec{v}_f.\vec{\nabla}\vec{v}_f =
    -\dfrac{1}{\rho_f} \vec{\nabla}p + \dfrac{\mu_f}{\rho_f} \nabla^2\vec{v}_f + \vec{g}$
  \end{block}
  
  \begin{block}{Desvantagens}
    \begin{itemize}
     \item Acoplamento da pressão e velocidade
     \item Exige elementos de ordem elevada
    \end{itemize}
  \end{block}

\end{frame}


\begin{frame}
  \frametitle{\subsecname}
  %-Esta formulação e uma alternativa a equação de Navier-Stoakes para encontrar o campo de velocidades do escoamento
  % Desta forma, consegue-se solucionar o sistema menos complexo, pois NS exige elementos de ordem superior para garantir a converg^encia
  % Isto permite uma solução mais simples e rápida
  
  \begin{block}{Equação da Vorticidade}
    \centering
    $\dfrac{\partial \vec{\omega}}{\partial t} + \vec{v}_f.\vec{\nabla}\vec{\omega} = \dfrac{\mu_f}{\rho_f} \nabla^2 \vec{\omega}$
  \end{block}
  
  \begin{block}{Equação da Corrente}
    \centering
    $\nabla^2\psi = -\omega_z$
  \end{block}
  
  \begin{block}{Velocidade}
    \centering
    $\vec{v}_f = \left(\dfrac{\partial \psi}{\partial y}, -\dfrac{\partial \psi}{\partial x} \right)$
  \end{block}

\end{frame}

%----------------------------------------------------------------------------------------------------------------------------------------------------
\subsection{Equação de Basset–Boussinesq–Oseen (BBO)}
\begin{frame}
  \frametitle{\subsecname}
  %-Esta equação e restrita para apenas valores de Re < 1
  
  \begin{block}{}
    Equação que representa as forças exercidas sobre as partículas.
    Sua expressão é a soma das forças separadamente.
  \end{block}
  
  \begin{block}{Equação de Basset–Boussinesq–Oseen}
    \centering
    $\vec{F}_{p} = \sum\vec{F} = \vec{F}_{grav} + \vec{F}_{drag} + \vec{F}_{lift} + \vec{F}_{mass}$
  \end{block}
  
  \begin{minipage}{.53\textwidth}
    \begin{block}{Restrição}
      A equação BBO é somente válida para Reynolds da partícula menores que 1.
      $Re_{p} < 1$
    \end{block}
    
  \end{minipage}
  \hfill
  \begin{minipage}{.41\textwidth}
    \begin{block}{Reynolds de Partícula}
      \centering
      $Re_{p} = \dfrac{\rho_p |\left(\vec{v}_{f} - \vec{v}_{p} \right)|_{max}\, d_{p}}{\mu_f}$
    \end{block}
    
  \end{minipage}
  
\end{frame}


\begin{frame}
  \frametitle{\subsecname}
  
  \begin{minipage}{.45\textwidth}
    \begin{block}{Força Gravitacional}
      \centering
      $\vec{F}_{grav} = m_p \vec{g}$
    \end{block}
    
    \begin{block}{Força de Arrasto}
      \centering
      $\vec{F}_{drag} = 3 \pi \mu_f d_p \left(\vec{v}_{f} - \vec{v}_{p} \right)$
    \end{block}
    
  \end{minipage}
  \hfill
  \begin{minipage}{.51\textwidth}
    \begin{block}{Força de Sustentação}
      \centering
      $\vec{F}_{lift} = 1.61 \mu_f d_p \left(\vec{v}_{f} - \vec{v}_{p} \right) \sqrt{{Re}_G}$
    \end{block}
    
    \begin{block}{Força de Massa Virtual}
      \centering
      $\vec{F}_{mass} = \dfrac{1}{2} \rho_f V_p \dfrac{d}{dt}\left(\vec{v}_{f} - \vec{v}_{p} \right)$
    \end{block}
    
  \end{minipage}
  
  \begin{block}{Reynolds de Cisalhamento}
    \centering
    $Re_G = \dfrac{d_p^2 \rho_f}{\mu_f} \nabla \vec{v}_f$
  \end{block}
  
\end{frame}
%----------------------------------------------------------------------------------------------------------------------------------------------------
\subsection{Sistema de Equações}
\begin{frame}
  \frametitle{Modelo Matemático}
  \begin{minipage}{.48\textwidth}
    \begin{block}{Equação de Vorticidade}
      \centering
      $\partfrac2{\omega_z}{t} + \vec{v}.\nabla\omega_z = \nu \nabla^2 \omega_z$
    \end{block}

    \begin{block}{Equação de Corrente}
      \centering
      $\nabla^2 \psi = -\omega_z$
    \end{block}
    
    \begin{block}{Equação BBO (\it{Basset–Boussinesq–Oseen})}
      \centering
      $\sum \vec{F}_p = \vec{F}_{drag} + \vec{F}_{grav} + \vec{F}_{etc}$
    \end{block}
  \end{minipage}
  \hfill
  \begin{minipage}{.48\textwidth}
    \begin{block}{Equações Auxiliares}
      \vspace*{-\baselineskip}\setlength\belowdisplayshortskip{0pt} % Fix display bug, empty header space
      \centering
      \begin{align*}
	\partfrac2{\psi}{y} &= v_x \\
	\partfrac2{\psi}{x} &= -v_y \\
	\omega_z &= \partfrac2{v_x}{y} - \partfrac2{v_y}{x} 
      \end{align*}
    \end{block}
    
    \begin{block}{Força de Arrasto (\it{Stoakes})}
      \centering
      $\vec{F}_{drag} = 3 \pi \mu d_p (\vec{v} - \vec{v}_p)$
    \end{block}
    
    \begin{block}{Força Gravitacional}
      \centering
      $\vec{F}_{grav} = \tfrac{\pi}{6} d_p \rho_p \vec{g}$
    \end{block}
  \end{minipage}
  
\end{frame}

% \begin{frame}
%   \frametitle{Modelo Matemático}
%   \begin{block}{Força de Arrasto (\it{Stoakes})}
%     \centering
%     $\vec{F}_{drag} = 3 \pi \mu d_p (\vec{v} - \vec{v}_p)$
%   \end{block}
%   \begin{block}{Força Gravitacional}
%     \centering
%     $\vec{F}_{grav} = \tfrac{\pi}{6} d_p \rho_p \vec{g}$
%   \end{block}
%   
%   Onde: $\omega_z$ é o campo de vorticidade, $\psi$ é o campo de correntes, $\vec{v}$ é o campo vetorial de velocidades, 
%   $\mu$ é a viscosidade dinâmica, $\nu$ é a viscosidade cinemática sobre o domínio da malha.
%   E as variáveis $d_p$ são o diâmetro, $\rho_p$ é a densidade e $\vec{F}_p$ é a força resultante de uma partícula.
% \end{frame}

%----------------------------------------------------------------------------------------------------------------------------------------------------
\subsection{Equações Matriciais}
\begin{frame}
  \frametitle{Matrizes dos Elementos Triangulares}
  \begin{minipage}{.25\textwidth}
    \centering
    \includegraphics[width=0.95\linewidth]{figure/element.pdf}
  \end{minipage}
  \hfill
  \begin{minipage}{.7\textwidth}
    \begin{block}{Coeficientes de Forma}
      \vspace*{-\baselineskip}\setlength\belowdisplayshortskip{0pt} % Fix display bug, empty header space
      \centering
      \begin{equation*}
      \mathbf{B} \left\{
	\begin{align*}
	  &b_i = y_j - y_k \\
	  &b_j = y_k - y_i \\
	  &b_k = y_i - y_j
	\end{align*} \right.\qquad
      \mathbf{C} \left\{
	\begin{align*}
	  &c_i = x_k - x_j \\
	  &c_j = x_i - x_k \\
	  &c_k = x_j - x_i
	\end{align*} \right.
      \end{equation*}
    \end{block}
  \end{minipage}
  
  \begin{minipage}[t]{.45\textwidth}
    \centering
    \begin{block}{Matriz de Gradiente (eixo x)}
      $\mathbf{G}_x = \frac{1}{6}
	\begin{bmatrix}
	  b_i & b_j & b_k \\
	  b_i & b_j & b_k \\
	  b_i & b_j & b_k
	\end{bmatrix}$
    \end{block}
  \end{minipage}
  \hfill
  \begin{minipage}[t]{.45\textwidth}
    \centering
    \begin{block}{Matriz de Gradiente (eixo y)}
      $\mathbf{G}_y = \frac{1}{4A}
	\begin{bmatrix}
	  c_i & c_j & c_k \\
	  c_i & c_j & c_k \\
	  c_i & c_j & c_k
	\end{bmatrix}$
    \end{block}
  \end{minipage}
  
  \begin{minipage}[t]{.63\textwidth}
    \centering
    \begin{block}{Matriz de Rigidez}
      \footnotesize
      $\mathbf{K} = \frac{1}{4A}
	\begin{bmatrix}
	  b_i^2 + c_i^2 & b_i b_j + c_i c_j & b_i b_k + c_i c_k \\
	  b_j b_i + c_j c_i & b_j^2 + c_j^2 & b_j b_k + c_j c_k \\
	  b_k b_i + c_k c_i & b_k b_j + c_k c_j & b_k^2 + c_k^2
	\end{bmatrix}$
    \end{block}
  \end{minipage}
  \hfill
  \begin{minipage}[t]{.33\textwidth}
    \centering
    \begin{block}{Matriz de Massa}
      $\mathbf{M} = \frac{A}{12}
	\begin{bmatrix}
	  2 & 1 & 1   \\
	  1 & 2 & 1   \\
	  1 & 1 & 2  
	\end{bmatrix}$
    \end{block}
  \end{minipage}
\end{frame}

\begin{frame}
  \frametitle{Equações Matriciais}
  
  \begin{figure}
    \stackunder{
      \includegraphics[width=0.4\linewidth]{figure/rotor_mesh.pdf}
    } {\raggedleft \tiny Malha gerada para um perfil de rotor.}
  \end{figure}
  
  \begin{minipage}{.63\textwidth}
    \centering
    \begin{block}{Vorticidade}
      \centering
      $\bigg(\dfrac{\mathbf{M}}{\Delta t} + \nu \mathbf{K} + \mathbf{v.G} \bigg)\omega_z^{n+1} = \dfrac{\mathbf{M}}{\Delta t} \omega_z^n$
    \end{block}
    
    \begin{block}{Corrente}
      \centering
      $\mathbf{K} \psi = \mathbf{M} \omega_z$
    \end{block}
  \end{minipage}
  \hfill
  \begin{minipage}{.33\textwidth}
    \centering
    \begin{block}{Auxiliares}
      \vspace*{-\baselineskip}\setlength\belowdisplayshortskip{0pt} % Fix display bug, empty header space
      \centering
      \begin{align*}
	\mathbf{M} v_x &= \mathbf{G}_y \psi \\
	\mathbf{M} v_y &=-\mathbf{G}_x \psi \\
	\mathbf{M} \omega_z &= \mathbf{G}_x v_y - \mathbf{G}_y v_x
      \end{align*}
    \end{block}
  \end{minipage}
  
%   \ \\
%   As demais equações de força são calculadas para cada partícula individulamente.
\end{frame}

%----------------------------------------------------------------------------------------------------------------------------------------------------
\section{Resultados Preliminares}
%----------------------------------------------------------------------------------------------------------------------------------------------------

\begin{frame}
  \begin{figure}
    \stackunder{
      \includegraphics[width=\linewidth]{figure/poiseuille_results.png}
    } {\raggedleft \tiny Perfil de velocidades no eixo x para um escoamento entre placas.}
  \end{figure}
  
  \begin{minipage}{.49\textwidth}
    \begin{figure}
      \includegraphics[width=\linewidth]{figure/particles_results_0.png}
    \end{figure}
  \end{minipage}
  \begin{minipage}{.49\textwidth}
    \begin{figure}
      \includegraphics[width=\linewidth]{figure/particles_results_1.png}
    \end{figure}
  \end{minipage}
  
  \tiny{Demonstração de partículas em movimento}
\end{frame}


\begin{frame}
  \begin{figure}
    \stackunder{
      \includegraphics[width=\linewidth]{figure/poiseuille_results.png}
    } {\raggedleft \tiny Perfil de velocidades no eixo x para um escoamento entre placas.}
  \end{figure}
  
  \begin{minipage}{.49\textwidth}
    \begin{figure}
      \includegraphics[width=\linewidth]{figure/particles_results_2.png}
    \end{figure}
  \end{minipage}
  \begin{minipage}{.49\textwidth}
    \begin{figure}
      \includegraphics[width=\linewidth]{figure/particles_results_3.png}
    \end{figure}
  \end{minipage}
  
  \tiny{Demonstração de partículas em movimento}
\end{frame}

%----------------------------------------------------------------------------------------------------------------------------------------------------
\section{Cronograma Futuro}
%----------------------------------------------------------------------------------------------------------------------------------------------------

\begin{frame}
  \frametitle{Atividades Concluídas e Previsão}
  \begin{figure}
    \includegraphics[width=\linewidth]{figure/Cronograma.png}
    \caption{Cronograma previsto atualizado.}
  \end{figure}
\end{frame}


%----------------------------------------------------------------------------------------------------------------------------------------------------
% Bibliografia
%----------------------------------------------------------------------------------------------------------------------------------------------------

% \begin{frame}
%   \frametitle{Bibliografia}
%   \footnotesize{
%     \todo[inline]{Fazer bibliografia}
%     \begin{thebibliography}{99} % Beamer does not support BibTeX so references must be inserted manually as below
% 
%       \bibitem[biezuner]{p1} R.J. Biezuner (2007)
%       \newblock Métodos Numéricos para Equações Parciais Elípticas
%       \newblock \emph{Notas de Aula}
% 
%       \bibitem[fortuna]{p1} A.O. Fortuna (2000)
%       \newblock Técnicas Computacionais para Dinâmica dos Fluidos: Conceitos Básicos e Aplicações
%       \newblock \emph{Edusp}
% 
%       \bibitem[leveque]{p1} R.J. LeVeque (2007)
%       \newblock Finite Difference Methods for Ordinary and Partial Differential Equations. Steady-State and Time-Dependant Problems
%       \newblock \emph{SIAM}
% 
%       \bibitem[biezuner]{p1} J.R. Rodrigues (2015)
%       \newblock Introdução à Simulação de Reservatórios Petrolíferos
%       \newblock \emph{Programa de Verão LNCC}
%     \end{thebibliography}
%   }
% \end{frame}

%----------------------------------------------------------------------------------------------------------------------------------------------------
% Agradecimentos
%----------------------------------------------------------------------------------------------------------------------------------------------------

\begin{frame}
  \frametitle{Agradecimentos}
  \centering
  \begin{tikzpicture}
    \node[inner sep=0cm] (gesar) at (0,0){
      \includegraphics[width=0.2\textwidth]{figure/gesar-logo-new.png}};
    \node[inner sep=0cm] (uerj) at (4,0){
      \includegraphics[width=0.2\textwidth]{figure/UERJ.png}};
    \node[inner sep=0cm] (fen) at (8,0){
      \includegraphics[width=0.2\textwidth]{figure/fen-new.png}};
  \end{tikzpicture}
  \Huge{\centerline{Muito Obrigado!}}
\end{frame}

%----------------------------------------------------------------------------------------

\end{document}
% Reduzir numero de slides, usar mais de uma figura por slide