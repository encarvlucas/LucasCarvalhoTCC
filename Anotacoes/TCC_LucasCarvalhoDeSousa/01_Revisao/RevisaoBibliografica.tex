\chapter{REVISÃO BIBLIOGRÁFICA}
\label{rev_bib}
\section{\textbf{Introdução}}
Nesta seção é apresentada a literatura utilizada, analizando-se as partes pertinentes ao trabalho realizado. Os principais tópicos de estudo foram sobre os temas de Método de Elementos Finitos, Escoamentos Multifásicos e Escoamentos Particulados em Turbomáquinas. 

\section{\textbf{Método de Elementos Finitos}}


\section{\textbf{Escoamentos Multifásicos}}
Escoamentos multifásicos são utilizados largamente na área da engenharia para uma diversidade de aplicações.
Estes ocorrem quando há o transporte de mais de uma substância em fases não miscigenadas.
Estas fases podem estar ou não no mesmo estado, subdividindo os tipos de escoamento multifásico no tipo de interação entre as fases, líquido-líquido, gás-líquido e sólido-líquido.

Dentro da engenharia mecânica pode-se verificar a grande importância destes escoamentos em casos como a extração de petróleo, onde é injetado um fluido e é captada uma mistura deste com o óleo bruto, e em trocadores de calor que possuem interação entre os fluidos.

Para os escoamentos particulados, sólido-líquido ou sólido-gás com pequenos sólidos chamados de partículas, pode-se notar sua importância até mesmo no transporte de dejetos, ná área de saneamento.%, com os chamados escoamentos de superfície livre.
Alguns exemplos na seção de mecânica incluem o transporte de vapor com condensado, formação de bolhas em bombas e sólidos precipitados.

Um dos primeiros trabalhos sobre este tipo de escoamento foi Baker et al. (1965)\cite{Baker-1965}, sobre o comportamento de escoamentos multifásicos em transportes verticais.
Estes usados bastante em trocadores de calor, como forma de melhorar sua eficiência.

%Um dos tipos de escoamentos multifásicos é o escoamento particulado, sólido-líquido ou sólido-gás, que é o foco deste trabalho.
Elghobashi et al. (1991)\cite{Elghobashi-1991} estuda o comportamento de escoamentos partículados demonstrando o efeito da turbulência na simulação de escoamentos multifásicos.

Como apresentado por Balachandar et al. (2010)\cite{Balachandar-2010}, os valores da fração de volume ocupada pela fase dispersa e a razão entre a massa da fase dispersa e a massa da fase líquida servem como indicadores do nível de interação entre as fases.
Para valores muito pequenos, o efeito dominante é do escoamento, portanto neste caso pode-se levar em conta apenas os efeitos do fluido sobre as partículas, chamado de \textit{one-way flow}.
Para casos com valores maiores, as partículas tomam um papel mais significativo no escoamento e é preciso fazer uma ligação recíproca entre os mesmos.
Portanto, recalcula-se o escoamento levando em conta os efeitos das partículas no fluido, conhecido como de \textit{two-way flow}.
E, finalmente, quando estes valores forem mais elevados a fase particulada toma um papel importante no comportamento e torna-se necessário considerar até os efeitos de outras partículas sobre cada uma delas, como colisão, aglomeração e quebra, denominado por Elghobashi et al. (1994)\cite{Elghobashi-1994} de \textit{four-way flow}.

Para a modelagem das forças atuando sobre cada partícula no escoamento foi utilizada a equação \textbf{Basset–Boussinesq–Oseen} (BBO), apresentada por Shao-Lee Soo et al. (1999)\cite{ShaoLeeSoo-1999}.
Estas equação é subdivida em várias forças atuantes, como a gravidade, arrasto, massa virtual, entre outras.
Estas equações possuem uma restrição para sua validade, podendo apenas serem aplicadas para casos com baixo número de Reynolds.


\section{\textbf{Escoamentos Particulados em Turbomáquinas}}
Dentro do ciclo de vida de uma turbomáquina, pode-se esperar um certo desgaste devido a pequenas partículas que se chocam contra as paredes durante o movimento do rotor.
Este desgate está ligado as propriedades do escoamento assim como das partículas.

Este efeito está presente até em turbinas de aeronaves, como estudado por Hussein et al. (1973)\cite{Hussein-1973}.
Este tipo de escoamento sólido-gás é verificado em locais com altos níveis de poluição.
Pode-se encontrar pequenas partículas sólidas presentes no ar ingerido por turbinas industriais e de aeronaves.
A sua presença causa um desgaste acelerado nas regiões radiais da turbina, como verificado também por Tabakoff et al. (1986)\cite{Tabakoff-1986}, que apresentou as regiões de maior colisão das partículas.
Estes trabalhos lidam diretamente com os efeitos da colisão das partículas, utilizando modelos que representam o comportamento delas após o choque.

Entrando na área de sólido-líquido, Uzol et al. (2002)\cite{Uzol-2002} fez um estudo mostranto o escoamento de uma turbomáquina utilizando partículas inseridas e um velocímetro de imagem para acompanhar seu trajeto.
Este trabalho não estuda o específicamente um escoamento particulado porém o utiliza como ferramenta para visualizar seu comportamento.

Em Ghenaiet et al. (2005)\cite{Ghenaiet-2005} é estudado com mais profundidade os efeitos dos escoamentos particulados em turbomáquinas com líquidos.
Neste caso, é estudado os efeitos da degradação causada por partículas de areia injeridas pelo escoamento na performance de uma turbomáquina axial.
Foi criado um modelo preditivo e demonstrada uma correlação entre o tamanho das partículas e a velocidade da degradação causada.

Porém a causa mais comum de erosão em impelidores de turbomáquinas trabalhando com líquidos é a cavitação.
A cavitação é a formação de bolhas de ar devido à uma queda local na pressão que logo após serem geradas implodem, gerando ondas de vibração e podendo danificar locais próximos.
A \ref{JAC-Pump} mostra os efeitos do desgaste causado pelo efeito da cavitação em uma bomba.
\begin{figure}[H]
    \centering
    \stackunder{
        \includegraphics[width=0.6\linewidth]{figures/thermalchgo-017w2.jpg}
    } {\raggedleft \scriptsize Fonte: John Anspach Consulting\cite{JAC}.}
    \caption{Erosão em um impelidor causada pela cavitação.}
    \label{JAC-Pump}
\end{figure}

Blake et al. (1987)\cite{Blake-1987} estudou o comportamento da cavitação tomando o ponto de vista das bolhas criadas como partículas no escoamento.
Através desta interpretação, pode-se realizar simulações dos efeitos da cavitação com modelos de escoamentos particulados.