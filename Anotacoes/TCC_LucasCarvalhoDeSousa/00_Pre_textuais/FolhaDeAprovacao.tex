\addtocounter{page}{+1}
\begin{center}

\authorName

\vspace{1cm}

\textbf{\mainTitle}

\end{center}

\vspace{.4cm}

\begin{flushright}
\small{\parbox{8cm}{
\singlespacing{Projeto Final apresentado a Faculdade de Engenharia da Universidade do Estado do Rio de Janeiro, para obtenção do grau de \linebreak bacharel em Engenharia Mecânica}.
% \singlespacing{Dissertação apresentada, como requisito\linebreak parcial para obtenção do título de Mestre em Ciências, ao Programa de Pós-Graduação em Engenharia Mecânica, da Universidade do Estado do Rio de Janeiro. Área de\linebreak concentração: Fenômenos de Transporte}.
}}
\end{flushright}

\vspace{.6cm}


% insira abaixo a data de sua defesa
% Caso não tenha defendido ainda, deixe em branco

\noindent Aprovado em: 25 de Junho de 2019 %29 de Maio de 2018

\noindent Banca Examinadora:


%
%
% Os professores da UERJ DEVEM ser citados primeiro, independente de quem seja o orientador.
%
%



\vspace{.7cm}

\begin{flushright}
\parbox{12cm}{

\singlespacing

\hrulefill \\

\vspace{-.4cm}
Prof. D.Sc. Gustavo R. Anjos - Orientador
\newline
Universidade Federal do Rio de Janeiro - UFRJ - COPPE
\vspace{.7cm}

\hrulefill \\

\vspace{-.4cm}
Prof. D.Sc. Norberto Mangiavacchi
\newline
Departamento de Engenharia Mecânica - UERJ
\vspace{.7cm}

\hrulefill \\

\vspace{-.4cm}
Prof. D.Sc. Fábio Pereira dos Santos
\newline
Universidade Federal do Rio de Janeiro - UFRJ - COPPE
\vspace{.7cm}

\hrulefill \\

\vspace{-.4cm}
D.Sc. Leon Matos Ribeiro de Lima
\newline
Eletronuclear
\vspace{.7cm}

% \hrulefill \\

% \vspace{-.4cm}
% Prof. Dr. Nome do Professor 5
% \newline
% Universidade Federal do Rio de Janeiro - UFRJ - COPPE
% \vspace{.7cm}

}
\end{flushright}
\vfill

\begin{center}
Rio de Janeiro\linebreak \curYear
\end{center}