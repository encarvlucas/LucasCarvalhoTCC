\noindent\textbf{INTRODUÇÃO}
\\

As turbomáquinas são equipamentos utilizados em larga escala por diversas áreas da engenharia.
Por isso, é de grande interesse estudar o comportamento dos escoamentos gerados por estas máquinas.
Em especial, busca-se estudar a trajetória de partículas presentes nestes escoamentos durante seu funcionamento.

Este trabalho visa criar uma biblioteca na linguagem de programação \textit{Python} de simulação de escoamentos particulados em diversas geometrias.
A solução dos escoamentos é calculada pela formulação da corrente-vorticidade utilizando o Método de Elementos Finitos para as variáveis espaciais e o Método de Diferenças Finitas para o termo temporal das equações.
Para a simulação do movimento das partículas também será usado o Método de Diferenças Finitas, aplicado a equação Basset–Boussinesq–Oseen para solução das forças aplicadas as partículas.

Serão gerados resultados para problemas conhecidos na literatura como forma de verificação dos cálculos.
Em seguida, serão realizadas simulações na geometria da palheta de um rotor para estudar o comportamento do escoamento e das partículas em turbomáquinas.
