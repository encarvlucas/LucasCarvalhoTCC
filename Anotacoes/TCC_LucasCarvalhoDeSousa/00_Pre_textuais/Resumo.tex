\begin{center}
\textbf{RESUMO}
\end{center}

%
% O resumo deve ser organizado em apenas um parágrafo mesmo.
% O número de folha é o número de páginas do PDF -2. Isto ocorre pois na versão final (capa dura) a capa é removida e as duas primeiras páginas são impressas em uma % folha apenas (frente e verso).
%

$\!$\\

\hspace{-1.3cm}\textbf{SOUSA}, Lucas Carvalho de. \textit{\mainTitle}. \numPages. Monografia (Projeto Final para Bacharelado em Engenharia Mecânica) - Faculdade de Engenharia, Universidade do Estado do Rio de Janeiro~(UERJ), Rio de Janeiro, \curYear.

\vspace{.2cm}

Este trabalho tem como objetivo criar uma biblioteca de simulação numérica de escoamentos particulados bidimensionais, em regime laminar, na linguagem Python e desenvolvida com o paradigma de programação orientada a objeto e analisar os resultados de casos aplicados à turbomáquinas.
Foi utilizado o Método de Elementos Finitos para resolver o sistema de equações obtido da formulação de corrente-vorticidade da equação de Navier-Stokes, tomada para um fluido incompressível e newtoniano.
Para a modelagem do comportamento das partículas, foi empregado o Método de Diferenças Finitas para solucionar a equação de Basset–Boussinesq–Oseen, considerando o tipo de interação \textit{one-way}, sem efeito das partículas sobre o escoamento.

\vspace{1cm}

\hspace{-1.3cm}Palavras-chave: Método de Elementos Finitos, Formulação Corrente-Vorticidade, Escoamento Multifásico, Escoamento Particulado.