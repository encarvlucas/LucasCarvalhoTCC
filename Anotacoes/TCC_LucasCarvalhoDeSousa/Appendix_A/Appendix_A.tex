\setcounter{section}{0}
\setcounter{chapter}{0}%
 \renewcommand{\theequation}{A.\arabic{equation}}    
  % redefine the command that creates the equation no.
% \renewcommand{\thechapter}{\arabic{chapter}}%
\noindent\textbf{APÊNDICE A - SOLUÇÃO NUMÉRICA DA EQUAÇÃO DO CALOR BIDIMENSIONAL E VALIDAÇÃO DO CÓDIGO}
$\!$\\

% Nesse apêndice são desenvolvidas e executadas duas formas de solucionar 
% numericamente equações diferenciais parciais que possuem o termo difusivo ($\nabla^2$).

O estudo das soluções numérica e analítica de equações diferenciais parciais foi essencial para o desenvolvimento do presente trabalho. O método adotado foi o segundo esquema de Douglas \cite{DOUGLAS} (também conhecido por \textit{Stabilizing Correction}) para solução das EDP's que modelam os mecanismos de reação-difusão presentes no capítulo [4]. Como motivação, foi considerada a equação de calor bidimensional, uma vez que ela configura uma equação parabólica utilizada para modelar problemas com dependência espacial através do termo difusivo ($\nabla^2$), presente nas dinâmicas estudadas neste projeto. O desenvolvimento do código foi em \textit{python}. A equação da temperatura, com as hipóteses abaixo: 












