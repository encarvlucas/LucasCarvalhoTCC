% Modelo de Dissertação em Latex para o PPG em Engenharia Mecânica da UERJ
% Este modelo foi adaptado da versão disponibilizada no site da Engenharia Elétrica da UERJ
% http://www.pel.uerj.br/publico/Modelo_LaTeX_Dissertacao_UERJ.rar
% http://www.pel.uerj.br/defesas/
%
% Utilizei o WinEdt 6.0 com Miktex 2.9
%
% Para gerar o PDF usei a opção PDFTeXfy com o documento [masterthesis.tex] aberto e em foco.
%
% Não consegui usar as figuras em EPS como o modelo original. Usei PNG e JPG sem problemas.
%
% Felipe M. - 20/06/2012
%
% 												Command             10pt    11pt    12pt
% 												\tiny               5       6       6
% 												\scriptsize         7       8       8
% 												\footnotesize       8       9       10
% 												\small              9       10      10.95
% 												\normalsize         10      10.95   12
% 												\large              12      12      14.4
% 												\Large              14.4    14.4    17.28
% 												\LARGE              17.28   17.28   20.74
% 												\huge               20.74   20.74   24.88
% 												\Huge               24.88   24.88   24.88


\documentclass[a4paper,12pt,oneside,openany]{uerj}
\usepackage[english,brazil]{babel}
% \usepackage[math]{iwona}
% \DeclareFontFamily{U}{futm}{}
% \DeclareFontShape{U}{futm}{m}{n}{
%   <-> fourier-bb % changed from .92 to 1
%   }{}
% \DeclareMathAlphabet{\mathbb}{U}{futm}{m}{n}
%\usepackage[latin1]{inputenc}
%\usepackage{fourier}
%\usepackage[utopia]{mathdesign}
%\usepackage{fouriernc}
%\usepackage{fontspec}
\usepackage{mathptmx} % Times New Roman

%\usepackage{Times}
% \usepackage[T1]{fontenc}
\usepackage{lmodern}
%\setmainfont{Times}
%\setmainfont{Arial}
\usepackage[utf8]{inputenc}
\usepackage{enumerate}
\usepackage{cite}
\usepackage{epsf,epsfig,psfig}
\usepackage{pagina}
\usepackage{indentfirst}
\usepackage{theorem}
\usepackage{fancyhdr}
\usepackage{setspace}
\usepackage{boxedminipage}
\usepackage{float}
%\usepackage[style=base]{caption}
\usepackage{subcaption}
%\captionsetup{compatibility=false}
%\usepackage{subcaption}
%\usepackage[caption=false]{subfig}
\usepackage{silence}
\WarningFilter{caption}{Unsupported document class}
\usepackage{makeidx}
\usepackage{amsmath}
\usepackage{amsfonts}
\usepackage{import}
\usepackage{mathtools}
\usepackage{geometry}
\usepackage{stackengine}
\usepackage[hidelinks]{hyperref}
\pdfstringdefDisableCommands{\let\uppercase\relax} % desabilitar alguns warnings e \uppercase

%------------------------------------------------
\usepackage{xargs}                      % Use more than one optional parameter in a new commands
\usepackage[pdftex,dvipsnames]{xcolor}  % Coloured text etc.
\usepackage[colorinlistoftodos,prependcaption,textsize=normalsize]{todonotes}
\newcommandx{\balao}[2][1=]{\todo[linecolor=red,backgroundcolor=red!25,bordercolor=red,#1]{#2}}
\newcommandx{\change}[2][1=]{\todo[linecolor=blue,backgroundcolor=blue!25,bordercolor=blue,#1]{#2}}
\newcommandx{\info}[2][1=]{\todo[linecolor=OliveGreen,backgroundcolor=OliveGreen!25,bordercolor=OliveGreen,#1]{#2}}
\newcommandx{\improvement}[2][1=]{\todo[linecolor=Plum,backgroundcolor=Plum!25,bordercolor=Plum,#1]{#2}}
\newcommandx{\thiswillnotshow}[2][1=]{\todo[disable,#1]{#2}}
\newcommandx{\authorName}{Lucas Carvalho de Sousa}
\newcommandx{\mainTitle}{Simulação Numérica De Escoamentos Dispersos Utilizando Método De Elementos Finitos}
\newcommandx{\curYear}{2019}
\newcommandx{\numPages}{xx f}
\renewcommand{\refeq}[1]{Eq. \ref{#1}}
\renewcommand{\eqref}[1]{(Eq. \ref{#1})}
%------------------------------------------------
%------------------------------------------------
\usepackage{geometry}
\usepackage{mathtools}
%------------------------------------------------

\makeindex


\newtheorem{deff}{Definição}[section]
\numberwithin{equation}{chapter}

\theoremstyle{plain}

\bibliographystyle{abnt-num}



\begin{document}

% \hypersetup{
%      colorlinks,
%     citecolor=black,
%     filecolor=black,
%     linkcolor=black,
%     urlcolor=black,
%     linktoc=all
% }

\thispagestyle{empty}\import{00_Pre_textuais/}{Capa}
\pagebreak\thispagestyle{empty}\import{00_Pre_textuais/}{FolhaDeRosto}
\pagebreak\thispagestyle{empty}\import{00_Pre_textuais/}{FichaCatalog}   
\pagebreak\thispagestyle{empty}\import{00_Pre_textuais/}{FolhaDeAprovacao}
\pagebreak\thispagestyle{empty}\import{00_Pre_textuais/}{Dedicatoria}
\pagebreak\thispagestyle{empty}\import{00_Pre_textuais/}{Agradecimento}
%\pagebreak\thispagestyle{empty}\import{00_Pre_textuais/}{Epigrafe}    % não coloquei epígrafe no meu trabalho, mas fica aqui a chamada comentada.
\pagebreak\thispagestyle{empty}\import{00_Pre_textuais/}{Resumo}
\pagebreak\thispagestyle{empty}\import{00_Pre_textuais/}{Abstract}

\fancypagestyle{plain}{
\fancyhf{} % clear all header and footer fields
\renewcommand{\headrulewidth}{0pt}
\renewcommand{\footrulewidth}{0pt}}
\pagestyle{plain}

\pagebreak

\def\listfigurename{LISTA DE FIGURAS}\listoffigures
\def\listtablename{LISTA DE TABELAS}\listoftables
\import{00_Pre_textuais/}{ListaDeSiglas.tex}    % não coloquei LISTA DE SIGLAS no meu trabalho, mas fica aqui a chamada comentada.
\def\contentsname{SUMÁRIO}\tableofcontents

\fancypagestyle{plain}{
\fancyhf{} % clear all header and footer fields
\fancyhead[R]{\thepage}
\setlength{\voffset}{-1cm}
\setlength{\headsep}{1cm}
\renewcommand{\headrulewidth}{0pt}
\renewcommand{\footrulewidth}{0pt}}

\pagestyle{plain}

\pagebreak
\addcontentsline{toc}{chapter}{\hspace{1.7cm}\bfseries INTRODUÇÃO}
\import{00_Pre_textuais/}{Introducao.tex}

\import{01_Cap1/}{RevisaoBibliografica.tex}

\import{02_Metodologia/}{Equacoes.tex}
\import{02_Metodologia/}{Modelagem.tex}

\import{03_Validacoes/}{Validacoes.tex}

% inserir demais capítulos aqui
% -----------------------------
% -----------------------------
% -----------------------------
% -----------------------------

\pagebreak
\addcontentsline{toc}{chapter}{\hspace{1.7cm}\bfseries CONCLUSÃO}
\import{10_Conclusao/}{conclusao.tex}

% \pagebreak
% \addcontentsline{toc}{chapter}{\hspace{1.7cm}\bfseries APÊNDICE A}
% \import{Appendix_A/}{Appendix_A.tex}

\pagebreak
\addcontentsline{toc}{chapter}{\hspace{1.7cm}\bfseries REFERÊNCIAS}
\def\bibname{REFERÊNCIAS}
% \pagebreak
% \addcontentsline{toc}{chapter}{\hspace{1.7cm}\bfseries ANEXO A}
% \import{Appendix_A/}{Appendix_A.tex}


% abaixo segue a chamada para o arquivo [.BIB]. Utilizei o programa JABREF para montar o arquivo com minhas referências.
\nocite{*}
\bibliography{dissertacao}
\bibliographystyle{unsrt}




% \printindex    %Removi o índice remissivo para a versão oficial do trabalho.


\end{document}