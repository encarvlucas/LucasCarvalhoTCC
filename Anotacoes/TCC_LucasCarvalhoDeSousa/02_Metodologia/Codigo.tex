%\chapter{METODOLOGIA}
%\label{metodologia}
%--------------------------------------------------------------
\chapter{\textbf{CÓDIGO NUMÉRICO}}
\label{sec_codigo}

%--------------------------------------------------------------
\section{\textbf{Introdução}}
O código deste trabalho foi feito completamente pelo autor na linguagem \textbf{\textit{Python 3.5}}, com auxílio das bibliotecas: \textit{NumPy}, para manipulação de dados e matrizes, \textit{SciPy} para solução de sistemas e matrizes esparsas, \textit{Matplotlib} para a visualização de gráficos e \href{https://github.com/nschloe/pygmsh}{\textit{pygmsh}} para importação da malha no formato \verb|.msh|.

Para a geração de malhas foi utilizado o \textit{software open source \textbf{Gmsh}}.
Nele é possível desenhar o perfil do domínio e gerar os nós e elementos de acordo com o refino desejado.
Também pode-se forçar diferentes tamanhos de elementos em regiões de maior interesse.

Para a vizualização de resultados foi utilizado o \textit{software open source \textbf{ParaView}}, pois este permite uma vizualização da evolução dos resultados, assim como diversas ferramentas para avaliação dos dados.

Todos os softwares de terceiros foram usados com permissão e licensas apropriadas.

%--------------------------------------------------------------
\section{\textbf{Estrutura do Código}}
O código foi planejado e estruturado com o paradigma de \textit{Orientação em Objeto} em mente.
Isto é, o código é centralizado em classes que possuem propiedades e métodos referentes a modelagem física do problema estudado.
Por exemplo, as classes \verb|TccLib.Mesh| e \verb|TccLib.Particle| representam os conceitos físicos de malha e partícula respectivamente.
Estas classes criam objetos que contém suas propiedades individuais, como as densidades $\rho_f$ e $\rho_p$, e permitem um melhor entendimento do fenômeno.

Este código foi pensado em ser utilizado como uma biblioteca pública com acesso liberado a qualquer um interessado \href{https://github.com/encarvlucas/LucasCarvalhoTCC}{(link do projeto no GitHub)}.
Por isso, as funções e métodos possuem diversas opções de chamadas.
Mais informações estão disponíveis através do comando interno: \begin{verbatim}
    help(arg)
\end{verbatim}
substituindo-se o argumento arg pela função desejada.

Demonstra-se a seguir um exemplo de uso da biblioca:
\begin{verbatim}
    # Importação da biblioteca
    import TccLib

    # Importação da malha ou coordenadas de uma nova
    mesh = TccLib.Mesh("arquivo_da_malha.msh")
    # ou mesh = TccLib.Mesh([coordenadas (x, y)]

    # Definição das condições de contorno
    mesh.new_boundary_condition("nome da propiedade",
                                [índices dos nós], 
                                [valor da condição no nó],
                                [1 para Dirichlet ou 0 para Neumann])


    # Chamada para a função de solução
    v_x, v_y = TccLib.solve_velocity_field(mesh)

    # Adição de partículas
    mesh.add_particle(propiedades da partícula)
    # Loop de movimentação das partículas
    for t in time_list:
        TccLib.move_particles(mesh, (v_x, v_y))
\end{verbatim}

Agora serão explorados cada etapa separadamente.

%--------------------------------------------------------------
\subsection{\textbf{Importação da Malha}}
O objeto de malha \verb|TccLib.Mesh| possui um construtor que aceita como parâmetros uma lista de coordenadas dos nós da malha ou o nome de um arquivo na extensão \verb|.msh| para importação.
Após a obtenção dos nós é utilizada a função \verb|scipy.spatial.Delaunay()|, que retorna uma lista com os conjuntos de índices dos nós por elemento.

São cadastradas as informações da malha criada no objeto, como as coordenadas dos nós nos eixos $x$ e $y$, o número de elementos e o número de nós da malha.
E é criado um dicionário vazio para receber as condições de contorno, onde suas chaves são nomes fixos para cada problema.

%--------------------------------------------------------------
\subsection{\textbf{Definição das Condições de Contorno}}
A classe de condições de contorno \verb|TccLib.BoundaryConditions| é uma classe privada, ou seja, o usuário não interaje com ela diretamente.
Para cadastrar novas condições de contorno é preciso utilizar o método dentro da classe \verb|TccLib.Mesh.new_boundary_condition()|.

O nome da propiedade varia de acordo com a função de solução.
Para o propósito deste trabalho, a solução do sistema corrente-vorticidade, são registradas as condições de contorno com nomes: \verb|"psi"| para os valores da corrente, \verb|"vel_x"| para os valores da velocidade do fluido no eixo x e  \verb|"vel_y"| para os valores da velocidade do fluido no eixo y.

Os valores da condição de contorno são passados em três listas de mesmo tamanho, onde a primeira contém os índices dos elementos, a segunda contém os valores da condição correspondentes ao índice associado e a terceira contém um valor lógico para a condição correspondentes ao índice, onde verdadeiro representa uma condição do tipo Dirichlet e falso representa o tipo Neumann.
Caso seja fornecido um valor numérico, ou booleano, ao invés da segunda e terceira lista respectivamente, será interpretado que todos os valores da lista substituida possuem este valor.

%--------------------------------------------------------------
\subsection{\textbf{Funções de Solução}}
Esta biblioteca foi projetada para adaptar diversas funções de solução para cada modelo físico desejado.
No momento da escrita, há duas funções de soluções implementadas no código:
\begin{itemize}
    \item Equação de Poisson (MEF) - \verb|TccLib.solve_poisson()|
    \item Sistema Corrente-Vorticidade (MEF) - \verb|TccLib.solve_velocity_field()|
    % \item Movimento de Partículas (MDF) - \verb|TccLib.move_particles()|
\end{itemize}

A solução dos problemas é realizada de forma semelhante.
Primeiramente são geradas as malhas globais do domínio da malha em uma função auxiliar \verb|TccLib.get_matrices()|:
\begin{verbatim}
    # Loop em cada elemento na lista da malha
    for elem in mesh.ien:
        x = mesh.x[elem] # = [x_i, x_j, x_k]
        y = mesh.y[elem] # = [y_i, y_j, y_k]

        # Criação das matrizes locais
        ...

        # Registro das matrizes locais nas matrizes globais
        for i in range(3):
            for j in range(3):
                kx_global[elem[i], elem[j]] += k_x[i][j]
                ky_global[elem[i], elem[j]] += k_y[i][j]
                m_global[elem[i], elem[j]] += m[i][j]
                gx_global[elem[i], elem[j]] += g_x[i][j]
                gy_global[elem[i], elem[j]] += g_y[i][j]
\end{verbatim}
que retorna as matrizes globais na ordem apresentada.

Em seguida, são montadas as matrizes principais para a solução algébrica:
\begin{equation}
    A_{n\times n}.x_{n} = b_{n}
\end{equation}
onde $n$ é o número de nós da malha.

Então são aplicadas as condições de contorno, utilizando a função privada \verb|TccLib.apply_boundary_conditions()|:
\begin{verbatim}
    # Iteração sobre a lista de nós com condição de contorno
    for coluna in no_list:
        if condição é Dirichlet:
            # Dirichlet
            for linha in lista_de_linhas:
                # Compensar linhas anuladas
                b[linha] -= A[linha, coluna] * condicao[coluna]
                # Zerar termos na matriz principal
                A[linha, coluna] = 0.
                A[coluna, linha] = 0.
            # Igualar valor da condição no nó a ela mesma
            A[coluna, coluna] = 1.
            b[coluna, 0] = condicao[coluna]

        else:
            # Neumann Treatment
            b[coluna, 0] += condicao[coluna]
\end{verbatim}

Finalmente, o sistema algébrico é solucionado pela função terceirizada da biblioteca de álgebra linear, otimizada, \verb|scipy.sparse.linalg.spsolve()|.

As funções de solução retornam listas com o valor da solução calculada para cada nó na ordem de índices registrada no objeto de malha.
Há métodos criados para auxiliar a visualização dos resultados no objeto de malha, como o método \verb|TccLib.Mesh.show_velocity_quiver()| que apresenta um gráfico com flechas apontando a direção do fluxo em cada nó e com tamanho proporcional a seu módulo.


%--------------------------------------------------------------
\subsection{\textbf{Movimentação das Partículas}}
No caso do problema multifásico de partículas inseridas em um fluido em movimento, há uma etapa de movimentação das partículas separada da função de solução principal.
Isto é devido ao fato do método de solução escolhido utilizar a implementação \textit{one-way} como apresentado por Crowe (2011)\cite{crowe}.
Neste tipo de solução o movimento do fluido, meio principal,  afeta as partículas, meio secundário, porém elas não causam reflexos no fluido.
Esta implementação foi escolhida devida a demanda computacional elevada da implementação \textit{two-way}, que requeriria solucionar o campo de velocidades (MEF) mais vezes por iteração.
Então as partículas são simuladas percorrendo um escoamento permanente, com o valor do campo de velocidades sido calculado anteriormente.

As partículas são adicionadas a malha através de um método próprio, que leva como parâmetros as propiedades físicas da partícula, sua posição e condições iniciais e um nome para registro, \verb|TccLib.Mesh.add_particle()|.

Em cada chamada da função \verb|TccLib.move_particles()| são calculdadas as forças aplicadas a cada partícula presente na malha e as forças são adicionadas a um dicionário com seus respectivos nomes como chaves.
Para se calcular as forças, são necessárias diversas informações, cujas são captadas nesta função, como a velocidade do escoamento na posição da partícula.
E o valor do número de Reynolds \eqref{reynolds_p} é calculado em cada chamada, para verificar se a simulação respeita a restrição apresentada em \ref{sec_eq_part}.
Em seguida é chamada uma função interna da classe \verb|TccLib.Particle| para executar o reposicionamento das partículas, esta calcula a as novas posições das partículas e as move, caso respeitem as condições de colisão.

Portanto, a função de movimentação das partículas é aplicada depois da solução do sistema.
A função de movimentação \verb|TccLib.move_particles()| não possui retorno, porém pode-se analisar seu resultado observando-se a propiedade \verb|TccLib.Particle.position_history| das partículas presentes na malha.
Esta propiedade é uma lista que contém as posições percorridas pela partícula, o que permite a visualização gráfica do percurso da mesma.