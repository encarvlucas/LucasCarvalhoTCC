\chapter{METODOLOGIA}
\label{metodologia}
\section{\textbf{Introdução}}
Este trabalho apresenta a modelagem de fluidos e partículas em um sistema de escoamento multifásico, portanto é preciso definir o que é um escoamento e quais as suas restrições para este trabalho.
Um escoamento é o movimento das moléculas de um fluido em conjunto.
As moléculas são tomadas como elementos infinitesimais, porém tratados de forma que não haja espaços vazios entre elas.
Isto permite que as propiedades do fluido sejam tratadas pontualmente, podendo variar por exemplo sua densidade ou velocidade de ponto a ponto.
Pode-se então modelar o comportamento destes escoamentos seguindo as equções de conservação:
\begin{itemize}
	\item \hyperref[cons_massa]{Conservação de Massa}
	\item \hyperref[cons_qmov]{Conservação de Quantidade de Movimento}
\end{itemize}
%--------------------------------------------------------------
\section{\textbf{Equações de Governo}}
\label{sec_eq_governo}
%--------------------------------------------------------------
\subsection{\textbf{Conservação de Massa}}
\label{sec_cons_massa}
O princípio de conservação de massa sem geração descreve que, dentro de um volume de controle, a soma da \textbf{taxa de acúmulo de massa dentro do volume} com \textbf{o fluxo de massa que atravessa a fronteira do volume} é nula\cite{pontes_norberto}.

O acúmulo de massa dentro do volume de controle é definido como:
\begin{equation}
    \int_{V_c}\dfrac{\partial}{\partial t} dm =  \int_{V_c}\dfrac{\partial}{\partial t} (\rho_f d V_c)
    \label{acumulo_massa}
\end{equation}
\begin{equation}
    m =  \rho_f V_c
    \label{densidade}
\end{equation}
onde $V_c$ é o volume de controle, $dm$ é o elemento infinitesimal de massa e $\rho_f$ é a massa específica do fluido.

Simplificando a equação de acúmulo de massa \eqref{acumulo_massa}, tomada para um volume de controle permanente, que não varia no tempo, tem-se:
\begin{equation}
    \int_{V_c}\dfrac{\partial}{\partial t} (\rho_f d V_c) = \int_{V_c}\dfrac{\partial \rho_f}{\partial t} d V_c + \int_{V_c} \rho_f \dfrac{\partial d V_c}{\partial t} = \int_{V_c}\dfrac{\partial \rho_f}{\partial t} d V_c
    \label{acumulo_massa_simp}
\end{equation}

E o fluxo de massa que atravessa a fronteira é retratado como:
\begin{equation}
    \oint_{S}\rho_f \vec{v}_f.\vec{n} dA
    \label{fluxo_massa}
\end{equation}
onde $S$ é a curva de contorno da fronteira do volume de controle, $\vec{v}_f$ é o campo de velocidades do fluido e $dA$ é o elemento infinitesimal de área da superfície de contorno do volume de controle.

A conservação de massa é então representada como:
\begin{equation}
    \int_{V_c}\dfrac{\partial \rho_f}{\partial t} d V_c + \oint_{S}\rho_f \vec{v}_f.\vec{n} dA = 0
    \label{cons_mass}
\end{equation}

Pode-se rescrever a equação de conservação de massa aplicando-se o \textit{Teorema de Gauss}\cite{pontes_norberto} na integral de superfície:
\begin{equation}
    \int_{V_c}\dfrac{\partial \rho_f}{\partial t} d V_c + \int_{V_c}\vec{\nabla}.(\rho_f \vec{v}_f) d V_c = 
    \int_{V_c}\left(\dfrac{\partial \rho_f}{\partial t} + \vec{\nabla}.(\rho_f \vec{v}_f) \right)d V_c = 0
    \label{cons_mass_int}
\end{equation}
obtendo-se a equação integral da conservação de massa, onde $\vec{\nabla}$ é o operador diferencial gradiente de componentes $\vec{\nabla}=\left(\tfrac{\partial}{\partial \hat{i}}, \tfrac{\partial}{\partial \hat{j}}, \tfrac{\partial}{\partial \hat{k}}\right)$..

Ao considerar a conservação do ponto de vista pontual, pode-se remover o termo integral e escrever a forma diferencial da conservação de massa:
\begin{equation}
    \dfrac{\partial \rho_f}{\partial t} + \vec{\nabla}.(\rho_f \vec{v}_f) = 0
    \label{cons_mass_dif}
\end{equation}

Esta equação é denominada \textit{Equação da Continuidade}, e pode ser reescrita como:
\begin{equation}
    \dfrac{\partial \rho_f}{\partial t} + \vec{v}_f.\vec{\nabla} \rho_f + \rho_f \vec{\nabla}.\vec{v}_f = 0
    \label{continuity}
\end{equation}

Tomando-se algumas hipósteses, é possível simplificar mais a equação.
Para um fluido incompressível, com massa específica invariante na posição e no tempo, a equação da continuidade pode ser escrita como:
\begin{equation}
    \rho_f \vec{\nabla}.\vec{v}_f = 0
    \label{continuity_mid}
\end{equation}

Como a massa específica do fluido não pode ser nula, tem-se:
\begin{equation}
    \vec{\nabla}.\vec{v}_f = 0
    \label{continuity_final}
\end{equation}
tomada para um \textbf{fluido incompressível}.


%--------------------------------------------------------------
\subsection{\textbf{Conservação de Quantidade de Movimento}}
\label{sec_cons_qmov}
A conservação de quantidade de movimento é similar a conservação de massa, porém é tomada como um termo vetorial.
Para o caso deste trabalho, é utilizada a versão linear da conservação de quantidade de movimento.
Portanto, tem-se que a conservação da quantidade de movimento determina que a \textbf{taxa de acúmulo de quantidade de movimento linear dentro do volume de controle} mais \textbf{o fluxo de quantidade de movimento linear que atravessa a fronteira do volume de controle} é igual a \textbf{soma das forças aplicadas à superficie da fronteira do volume de controle e as forças do volume}.

A definição da taxa de acúmulo de quantidade de movimento linear dentro do volume de controle é:
\begin{equation}
    \int_{V_c}\dfrac{\partial}{\partial t} (\rho_f \vec{v}_f) d V_c
    \label{acumulo_qmov}
\end{equation}

E o fluxo de quantidade de movimento que atravessa a fronteira é retratado como:
\begin{equation}
    \oint_{S}\rho_f \vec{v}_f \vec{v}_f.\vec{n} dA
    \label{fluxo_qmov}
\end{equation}

As forças aplicadas à superfície da fronteira do volume de controle é:
\begin{equation}
    \oint_{S}\boldsymbol{\sigma}.\vec{n} dA
    \label{result_sup_qmov}
\end{equation}
onde $\boldsymbol{\sigma}$ é o tensor de tensões.

E as forças de volume são, tomada apenas a força gravitacional:
\begin{equation}
    \int_{V_c}\rho_f \vec{g} dV_c
    \label{result_vol_qmov}
\end{equation}
onde $\vec{g}$ é a aceleração gravitacional presente.

Montando-se a equação, a conservação de quantidade de movimento é representada como:
\begin{equation}
    \int_{V_c}\dfrac{\partial}{\partial t} (\rho_f \vec{v}_f) d V_c + 
    \oint_{S}\rho_f \vec{v}_f \vec{v}_f.\vec{n} dA =
    \oint_{S}\boldsymbol{\sigma}.\vec{n} dA +
    \int_{V_c}\rho_f \vec{g} dV_c
    \label{cons_qmov}
\end{equation}

Novamente, aplica-se o \textit{Teorema de Gauss} nas integrais de superfície e extrai-se a forma integral da equação da conservação de quantidade de movimento:
\begin{equation}
    \int_{V_c}\dfrac{\partial}{\partial t} (\rho_f \vec{v}_f) d V_c + 
    \int_{V_c}\vec{\nabla}.(\rho_f \vec{v}_f.\vec{v}_f) dV_c =
    \int_{V_c}\vec{\nabla}.\boldsymbol{\sigma} dV_c +
    \int_{V_c}\rho_f \vec{g} dV_c
\end{equation}

Simplificando:
\begin{equation}
    \int_{V_c} \left(
	    \dfrac{\partial}{\partial t} (\rho_f \vec{v}_f) + 
	    \vec{\nabla}.(\rho_f \vec{v}_f.\vec{v}_f) -
	    \vec{\nabla}.\boldsymbol{\sigma} -
	    \rho_f \vec{g}
    \right) = 0
    \label{cons_qmov_int}
\end{equation}

Novamente, considerando-se a conservação do ponto de vista pontual, remove-se o termo integral para escrever a forma diferencial da conservação de quantidade de movimento:
\begin{equation}
    \dfrac{\partial}{\partial t} (\rho_f \vec{v}_f) + 
    \vec{\nabla}.(\rho_f \vec{v}_f.\vec{v}_f) =
    \vec{\nabla}.\boldsymbol{\sigma} +
    \rho_f \vec{g}
    \label{cons_qmov_dif}
\end{equation}

Continuando a desenvolver a equação:
\begin{equation}
    \dfrac{\partial}{\partial t} (\rho_f \vec{v}_f) + 
    \vec{\nabla}(\rho_f \vec{v}_f.\vec{v}_f) =
    \rho_f \dfrac{\partial \vec{v}_f}{\partial t} + 
    \vec{v}_f \dfrac{\partial \rho_f}{\partial t} + 
    \rho_f \vec{v}_f.\vec{\nabla}.\vec{v}_f +
    \vec{v}_f.\vec{\nabla}.(\rho_f \vec{v}_f)
    \label{cons_qmov_ini}
\end{equation}
\begin{equation}
    \rho_f \left(
    	\dfrac{\partial \vec{v}_f}{\partial t} +
    	\vec{v}_f.\vec{\nabla}.\vec{v}_f 
	\right) +
	\vec{v}_f \left(
    	\dfrac{\partial \rho_f}{\partial t} +
    	\vec{\nabla}.(\rho_f \vec{v}_f)
	\right)
    \label{cons_qmov_mid}
\end{equation}

Novamente, é tomanda a hipóstese de um fluido incompressível.
Portanto, a equação pode ser simplificada para ser escrita como:
\begin{equation}
    \rho_f \left(
    	\dfrac{\partial \vec{v}_f}{\partial t} + 
    	\vec{v}_f.\vec{\nabla}.\vec{v}_f
	\right) =
    \vec{\nabla}.\boldsymbol{\sigma} +
    \rho_f \vec{g}
    \label{cons_qmov_fin}
\end{equation}

Reescrevendo o tensor de tensões como uma soma de dois tensores, e o substituindo na equação, tem-se:
\begin{equation}
    \boldsymbol{\sigma} = -p\mathbf{I} + \boldsymbol{\tau}
    \label{sigma}
\end{equation}
\begin{equation}
    \rho_f \left(
    	\dfrac{\partial \vec{v}_f}{\partial t} + 
    	\vec{v}_f.\vec{\nabla}.\vec{v}_f
	\right) =
    -\vec{\nabla}p +
    \vec{\nabla}.\boldsymbol{\tau} +
    \rho_f \vec{g}
    \label{cons_qmov_tensor}
\end{equation}
onde p é o campo de pressões no fluido, $\mathbf{I}$ é a matriz de indentidade e $\boldsymbol{\tau}$ é o tensor de tensões viscosas.

Novamente é necessário fazer uma hipótese para este escoamento, para que seja possível definir as forças atuantes no fluido. 
O tensor de tensões viscosas $\boldsymbol{\tau}$ está relacionado as propiedades do fluido, podendo ser definido matematicamente para um fluido homogêneo, isotrópico e newtoniano, como:
\begin{equation}
    \boldsymbol{\tau} = \mu_f\left(\vec{\nabla}.\vec{v}_f + \left(\vec{\nabla}.\vec{v}_f \right)^T \right)
    \label{tau}
\end{equation}
onde $\mu_f$ é a viscosidade dinâmica do fluido.

Substituindo a definição de $\boldsymbol{\tau}$ na \refeq{cons_qmov_tensor}, obtem-se:
\begin{equation}
    \rho_f \left(
    	\dfrac{\partial \vec{v}_f}{\partial t} + 
    	\vec{v}_f.\vec{\nabla}.\vec{v}_f
	\right) =
    -\vec{\nabla}p +
    \vec{\nabla}.\left(\mu_f\left(\vec{\nabla}.\vec{v}_f + \left(\vec{\nabla}.\vec{v}_f \right)^T \right)\right) +
    \rho_f \vec{g}
\end{equation}

Assumindo-se que a viscosidade dinâmica é constante para todo o fluido:
\begin{equation}
    \rho_f \left(
    	\dfrac{\partial \vec{v}_f}{\partial t} + 
    	\vec{v}_f.\vec{\nabla}.\vec{v}_f
	\right) =
    -\vec{\nabla}p +
    \mu_f \vec{\nabla}.\left(\vec{\nabla}.\vec{v}_f + \left(\vec{\nabla}.\vec{v}_f \right)^T \right) +
    \rho_f \vec{g}
\end{equation}
\begin{equation}
    \rho_f \left(
    	\dfrac{\partial \vec{v}_f}{\partial t} + 
    	\vec{v}_f.\vec{\nabla}.\vec{v}_f
	\right) =
    -\vec{\nabla}p +
    \mu_f \left(\nabla^2\vec{v}_f + \vec{\nabla}.\left(\vec{\nabla}.\vec{v}_f \right)^T \right) +
    \rho_f \vec{g}
    \label{tau_last}
\end{equation}

Utilizando o que foi obtido na Equação de Continuidade \refeq{continuity_final} pode-se substituir na equação \refeq{tau_last}:
\begin{equation}
    \rho_f \left(
    	\dfrac{\partial \vec{v}_f}{\partial t} + 
    	\vec{v}_f.\vec{\nabla}.\vec{v}_f
	\right) =
    -\vec{\nabla}p +
    \mu_f \nabla^2\vec{v}_f +
    \rho_f \vec{g}
    \label{tau_simple}
\end{equation}

Dividindo-se todos os termos pela massa específica pode-se reescrever a equação \refeq{tau_simple}, obtem-se então a forma simplificada da \textit{Equação de Navier-Stoakes} para fluidos neutonianos incompressíveis e com viscosidade constante:
\begin{equation}
	\dfrac{\partial \vec{v}_f}{\partial t} + 
	\vec{v}_f.\vec{\nabla}.\vec{v}_f =
    -\dfrac{1}{\rho_f} \vec{\nabla}p +
    \dfrac{\mu_f}{\rho_f} \nabla^2\vec{v}_f +
    \vec{g}
    \label{navier}
\end{equation}

%--------------------------------------------------------------
\subsection{\textbf{Formulação Corrente-Vorticidade}}
\label{sec_corr_vort}
A formulação de corrente-vorticidade é um sistema de equações que providencia um método alternativo de se calcular as propiedades de um escoamento sem solucionar diretamente a equação de Navier-Stoakes \eqref{navier}.
Isto permite simplificar a solução do problema, pois a Equação de Navier-Stoakes possui um forte acoplamento entre o campo de pressões e o campo de velocidades.

Para isso, é utilizada a seguinte identidade vetorial:
\begin{equation}
    \vec{v}.\vec{\nabla}.\vec{v} = \vec{\nabla} \dfrac{v^2}{2} - \vec{v}\times\vec{\nabla}\times\vec{v}
    \label{ident_vetor} 
\end{equation}

Substituindo na \refeq{navier}:
\begin{equation}
	\dfrac{\partial \vec{v}_f}{\partial t} + 
	\vec{\nabla} \dfrac{v^2_f}{2} -
	\vec{v}_f\times\vec{\nabla}\times\vec{v}_f =
    -\dfrac{1}{\rho_f} \vec{\nabla}p +
    \dfrac{\mu_f}{\rho_f} \nabla^2\vec{v}_f +
    \vec{g}
\end{equation}

Em seguida, aplica-se o operador rotacional nos dois lados da equação:
\begin{equation}
	\vec{\nabla}\times\dfrac{\partial \vec{v}_f}{\partial t} + 
	\vec{\nabla}\times\vec{\nabla} \dfrac{v^2_f}{2} -
	\vec{\nabla}\times\vec{v}_f\times\vec{\nabla}\times\vec{v}_f =
    -\vec{\nabla}\times\dfrac{1}{\rho_f} \vec{\nabla}p +
    \vec{\nabla}\times\dfrac{\mu_f}{\rho_f} \nabla^2\vec{v}_f +
    \vec{\nabla}\times\vec{g}
\end{equation}

Simplifica-se então a equação, pois os termos que possuem o operador gradiente são anulados, pois o rotacional do gradiente de um escalar é zero.
O termo gravitacional também é anulado já que a derivada da constante $\vec{g}$ é zero.
Fica-se com a seguinte equação:
\begin{equation}
	\dfrac{\partial}{\partial t}(\vec{\nabla}\times\vec{v}_f)-
	\vec{\nabla}\times\vec{v}_f\times\vec{\nabla}\times\vec{v}_f =
    \dfrac{\mu_f}{\rho_f} \nabla^2 (\vec{\nabla}\times\vec{v}_f)
\end{equation}

Define-se o vetor $\vec{\omega}$ como \textit{vorticidade}, onde $\vec{\omega}=\vec{\nabla}\times\vec{v}_f$.
Substitui-se na equação:
\begin{equation}
	\dfrac{\partial \vec{\omega}}{\partial t} -
	\vec{\nabla}\times\vec{v}_f\times\vec{\omega} =
    \dfrac{\mu_f}{\rho_f} \nabla^2 \vec{\omega}
\end{equation}

Rearranjando-se as operações vetoriais, tem-se:
\begin{equation}
	\dfrac{\partial \vec{\omega}}{\partial t} +
	\vec{v}_f.\vec{\nabla}.\vec{\omega} -
	\vec{\omega}.\vec{\nabla}.\vec{v}_f =
    \dfrac{\mu_f}{\rho_f} \nabla^2 \vec{\omega}
\end{equation}

Como a vorticidade é perpendicular ao vetor velocidade, para escoamentos bidimensionais pode-se anular o produto $\vec{\omega}.\vec{\nabla}.\vec{v}_f$\cite{pontes_norberto}.
Então obtem-se a \textit{Equação da Vorticidade} para escoamentos de fluidos neutonianos incompressíveis:
\begin{equation}
	\dfrac{\partial \vec{\omega}}{\partial t} +
	\vec{v}_f.\vec{\nabla}.\vec{\omega} =
    \dfrac{\mu_f}{\rho_f} \nabla^2 \vec{\omega}
    \label{vorticity}
\end{equation}

Para escoamentos permanentes bidimensionais de fluidos incompressíveis, a velocidade é calculada pela vazão volumétrica.
Portanto, a velocidade pode ser substituída por um escalar $\psi$, conhecido como \textit{função corrente}.
A relação entre a função corrente e o campo de velocidades do fluido é obtida através da manipulação da equação da continuidade \eqref{continuity}, dada como:
\begin{equation}
	\dfrac{\partial v_x}{\partial x} +
	\dfrac{\partial v_y}{\partial y} =
	0
\end{equation}
onde $v_x$ é a componente do campo de velocidades do fluido no eixo $x$ e $v_y$ é a componente do campo de velocidades do fluido no eixo $y$.

E a relação entre elas é apresentada em seguida:
\begin{equation}
	v_x = \dfrac{\partial \psi}{\partial y}
	\label{corr_x}
\end{equation}
\begin{equation}
	v_y = -\dfrac{\partial \psi}{\partial x}
	\label{corr_y}
\end{equation}

Enquanto a relação entre a função corrente e a vorticidade é:
\begin{equation}
	(\vec{\nabla}\times\vec{v}_f)_z =
	\dfrac{\partial v_y}{\partial x} -
	\dfrac{\partial v_x}{\partial y}
\end{equation}

Substituindo os termos de velocidade pelas equações \refeq{corr_x} e \refeq{corr_y}:
\begin{equation}
	\omega_z =
	-\dfrac{\partial}{\partial x}\dfrac{\partial \psi}{\partial x} -
	\dfrac{\partial}{\partial y}\dfrac{\partial \psi}{\partial y}
\end{equation}
\begin{equation}
	\omega_z =
	-\nabla^2\psi
	\label{corrente_vorticidade}
\end{equation}

Então o sistema de corrente-vorticidade é apresentado como:
\begin{equation}
	\dfrac{\partial \vec{\omega}}{\partial t} +
	\vec{v}_f.\vec{\nabla}.\vec{\omega} =
    \dfrac{\mu_f}{\rho_f} \nabla^2 \vec{\omega}
\end{equation}
\begin{equation}
	\nabla^2\psi =
	-\omega_z
\end{equation}
\begin{equation}
	\vec{v}_f = \left(\dfrac{\partial \psi}{\partial y},
	-\dfrac{\partial \psi}{\partial x} \right)
\end{equation}


%--------------------------------------------------------------
\subsection{\textbf{Forças Exercidas em Partículas}}
\label{sec_eq_part}
A principal equação que governa o comportamento do movimento das partículas é obtida a partir da 2ª Lei de Newton\cite{crowe}.
\begin{equation}
    \sum\vec{F}_{p} = \dfrac{d(m_p\vec{v}_p)}{dt}
    \label{newton} 
\end{equation}


\begin{equation}
    \vec{F}_{mass} = \dfrac{\pi}{12} \rho_{f} d_{p}^3 \dfrac{d}{dt} \left(\vec{v}_{f} - \vec{v}_{p} \right) 
    \label{added_mass} 
\end{equation}