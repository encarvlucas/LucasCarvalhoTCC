\chapter{METODOLOGIA}
\label{metodologia}
\section{\textbf{Introdução}}
Este trabalho apresenta a modelagem de fluidos e partículas em um sistema de escoamento multifásico, portanto é preciso definir o que é um escoamento e quais as suas restrições para este trabalho.
Um escoamento é o movimento das moléculas de um fluido em conjunto.
As moléculas são tomadas como elementos infinitesimais, porém tratados de forma que não haja espaços vazios entre elas.
Isto permite que as propiedades do fluido sejam tratadas pontualmente, podendo variar por exemplo sua densidade ou velocidade de ponto a ponto.
Pode-se então modelar o comportamento destes escoamentos seguindo as equções de conservação:
\begin{itemize}
	\item \hyperref[cons_massa]{Conservação de Massa}
	\item \hyperref[cons_qmov]{Conservação de Quantidade de Movimento}
\end{itemize}
%--------------------------------------------------------------
\section{\textbf{Equações de Governo}}
\label{sec_eq_governo}
%--------------------------------------------------------------
\subsection{\textbf{Conservação de Massa}}
\label{sec_cons_massa}
O princípio de conservação de massa sem geração descreve que, dentro de um volume de controle, a soma da \textbf{taxa de acúmulo de massa dentro do volume} com \textbf{o fluxo de massa que atravessa a fronteira do volume} é nula\cite{pontes_norberto}.

O acúmulo de massa dentro do volume de controle é definido como:
\begin{equation}
    \int_{V_c}\dfrac{\partial}{\partial t} dm =  \int_{V_c}\dfrac{\partial}{\partial t} (\rho_f d V_c)
    \label{acumulo_massa}
\end{equation}
\begin{equation}
    m =  \rho_f V_c
    \label{densidade}
\end{equation}
onde $V_c$ é o volume de controle, $dm$ é o elemento infinitesimal de massa e $\rho_f$ é a massa específica do fluido.

Simplificando a equação de acúmulo de massa \eqref{acumulo_massa}, tomada para um volume de controle permanente, que não varia no tempo, tem-se:
\begin{equation}
    \int_{V_c}\dfrac{\partial}{\partial t} (\rho_f d V_c) = \int_{V_c}\dfrac{\partial \rho_f}{\partial t} d V_c + \int_{V_c} \rho_f \dfrac{\partial d V_c}{\partial t} = \int_{V_c}\dfrac{\partial \rho_f}{\partial t} d V_c
    \label{acumulo_massa_simp}
\end{equation}

E o fluxo de massa que atravessa a fronteira é retratado como:
\begin{equation}
    \oint_{S}\rho_f \vec{v}_f.\vec{n} dA
    \label{fluxo_massa}
\end{equation}
onde $S$ é a curva de contorno da fronteira do volume de controle, $\vec{v}_f$ é o campo de velocidades do fluido e $dA$ é o elemento infinitesimal de área da superfície de contorno do volume de controle.

A conservação de massa é então representada como:
\begin{equation}
    \int_{V_c}\dfrac{\partial \rho_f}{\partial t} d V_c + \oint_{S}\rho_f \vec{v}_f.\vec{n} dA = 0
    \label{cons_mass}
\end{equation}

Pode-se rescrever a equação de conservação de massa aplicando-se o \textit{Teorema de Gauss}\cite{pontes_norberto} na integral de superfície:
\begin{equation}
    \int_{V_c}\dfrac{\partial \rho_f}{\partial t} d V_c + \int_{V_c}\vec{\nabla}.(\rho_f \vec{v}_f) d V_c = 
    \int_{V_c}\left(\dfrac{\partial \rho_f}{\partial t} + \vec{\nabla}.(\rho_f \vec{v}_f) \right)d V_c = 0
    \label{cons_mass_int}
\end{equation}
obtendo-se a equação integral da conservação de massa, onde $\vec{\nabla}$ é o operador diferencial \textit{nabla} de componentes $\vec{\nabla}=\left(\tfrac{\partial}{\partial \hat{i}}, \tfrac{\partial}{\partial \hat{j}}, \tfrac{\partial}{\partial \hat{k}}\right)$..

Ao considerar a conservação do ponto de vista pontual, pode-se remover o termo integral e escrever a forma diferencial da conservação de massa:
\begin{equation}
    \dfrac{\partial \rho_f}{\partial t} + \vec{\nabla}.(\rho_f \vec{v}_f) = 0
    \label{cons_mass_dif}
\end{equation}

Esta equação é denominada \textit{Equação da Continuidade}, e pode ser reescrita como:
\begin{equation}
    \dfrac{\partial \rho_f}{\partial t} + \vec{v}_f.\vec{\nabla} \rho_f + \rho_f \vec{\nabla}.\vec{v}_f = 0
    \label{continuity}
\end{equation}

Tomando-se algumas hipósteses, é possível simplificar mais a equação.
Para um fluido incompressível, com massa específica invariante na posição e no tempo, a equação da continuidade pode ser escrita como:
\begin{equation}
    \rho_f \vec{\nabla}.\vec{v}_f = 0
    \label{continuity_mid}
\end{equation}

Como a massa específica do fluido não pode ser nula, tem-se:
\begin{equation}
    \vec{\nabla}.\vec{v}_f = 0
    \label{continuity_final}
\end{equation}
tomada para um \textbf{fluido incompressível}.


%--------------------------------------------------------------
\subsection{\textbf{Conservação de Quantidade de Movimento}}
\label{sec_cons_qmov}
A conservação de quantidade de movimento é similar a conservação de massa, porém é tomada como um termo vetorial.
Para o caso deste trabalho, é utilizada a versão linear da conservação de quantidade de movimento.
Portanto, tem-se que a conservação da quantidade de movimento determina que a \textbf{taxa de acúmulo de quantidade de movimento linear dentro do volume de controle} mais \textbf{o fluxo de quantidade de movimento linear que atravessa a fronteira do volume de controle} é igual a \textbf{soma das forças aplicadas à superficie da fronteira do volume de controle e as forças do volume}.

A definição da taxa de acúmulo de quantidade de movimento linear dentro do volume de controle é:
\begin{equation}
    \int_{V_c}\dfrac{\partial}{\partial t} (\rho_f \vec{v}_f) d V_c
    \label{acumulo_qmov}
\end{equation}

E o fluxo de quantidade de movimento que atravessa a fronteira é retratado como:
\begin{equation}
    \oint_{S}\rho_f \vec{v}_f \vec{v}_f.\vec{n} dA
    \label{fluxo_qmov}
\end{equation}

As forças aplicadas à superfície da fronteira do volume de controle é:
\begin{equation}
    \oint_{S}\boldsymbol{\sigma}.\vec{n} dA
    \label{result_sup_qmov}
\end{equation}
onde $\boldsymbol{\sigma}$ é o tensor de tensões superficiais.

E as forças de volume são, tomada apenas a força gravitacional:
\begin{equation}
    \int_{V_c}\rho_f \vec{g} dV_c
    \label{result_vol_qmov}
\end{equation}
onde $\vec{g}$ é a aceleração gravitacional presente.

Montando-se a equação, a conservação de quantidade de movimento é representada como:
\begin{equation}
    \int_{V_c}\dfrac{\partial}{\partial t} (\rho_f \vec{v}_f) d V_c + 
    \oint_{S}\rho_f \vec{v}_f \vec{v}_f.\vec{n} dA =
    \oint_{S}\boldsymbol{\sigma}.\vec{n} dA +
    \int_{V_c}\rho_f \vec{g} dV_c
    \label{cons_qmov}
\end{equation}

Novamente, aplica-se o \textit{Teorema de Gauss} nas integrais de superfície:
\begin{equation}
    \int_{V_c}\dfrac{\partial}{\partial t} (\rho_f \vec{v}_f) d V_c + 
    \int_{V_c}\vec{\nabla}.(\rho_f \vec{v}_f.\vec{v}_f) dV_c =
    \int_{V_c}\vec{\nabla}.\boldsymbol{\sigma} dV_c +
    \int_{V_c}\rho_f \vec{g} dV_c
    \label{cons_qmov}
\end{equation}

Simplificando:
\begin{equation}
    \int_{V_c} \left(
	    \dfrac{\partial}{\partial t} (\rho_f \vec{v}_f) + 
	    \vec{\nabla}.(\rho_f \vec{v}_f.\vec{v}_f) -
	    \vec{\nabla}.\boldsymbol{\sigma} -
	    \rho_f \vec{g}
    \right) = 0
    \label{cons_qmov}
\end{equation}

\begin{equation}
    \vec{F}_{mass} = \dfrac{\pi}{12} \rho_{f} d_{p}^3 \dfrac{d}{dt} \left(\vec{v}_{f} - \vec{v}_{p} \right) 
    \label{added_mass} 
\end{equation}
